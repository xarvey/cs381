%%%%%%%%%%%%%%%%%%%%%%%%%%%%%%%%%%%%%%%%%
% Programming/Coding Assignment
% LaTeX Template
%
% This template has been downloaded from:
% http://www.latextemplates.com
%
% Original author:
% Ted Pavlic (http://www.tedpavlic.com)
%
% Note:
% The \lipsum[#] commands throughout this template generate dummy text
% to fill the template out. These commands should all be removed when 
% writing assignment content.
%
% This template uses a Perl script as an example snippet of code, most other
% languages are also usable. Configure them in the "CODE INCLUSION 
% CONFIGURATION" section.
%
%%%%%%%%%%%%%%%%%%%%%%%%%%%%%%%%%%%%%%%%%

%----------------------------------------------------------------------------------------
%	PACKAGES AND OTHER DOCUMENT CONFIGURATIONS
%----------------------------------------------------------------------------------------

\documentclass{article}

\usepackage{fancyhdr} % Required for custom headers
\usepackage{lastpage} % Required to determine the last page for the footer
\usepackage{extramarks} % Required for headers and footers
\usepackage[usenames,dvipsnames]{color} % Required for custom colors
\usepackage{graphicx} % Required to insert images
\usepackage{listings} % Required for insertion of code
\usepackage{courier} % Required for the courier font
\usepackage{lipsum} % Used for inserting dummy 'Lorem ipsum' text into the template
\usepackage{url}
\usepackage{mathtools}
\DeclarePairedDelimiter{\ceil}{\lceil}{\rceil}


% Margins
\topmargin=-0.45in
\evensidemargin=0in
\oddsidemargin=0in
\textwidth=6.5in
\textheight=9.0in
\headsep=0.25in

\linespread{1.1} % Line spacing

% Set up the header and footer
\pagestyle{fancy}
\lhead{\hmwkAuthorName} % Top left header
\chead{\hmwkClass\ (\hmwkClassInstructor\ \hmwkClassTime): \hmwkTitle} % Top center head
\rhead{\firstxmark} % Top right header
\lfoot{\lastxmark} % Bottom left footer
\cfoot{} % Bottom center footer
\rfoot{Page\ \thepage\ of\ \protect\pageref{LastPage}} % Bottom right footer
\renewcommand\headrulewidth{0.4pt} % Size of the header rule
\renewcommand\footrulewidth{0.4pt} % Size of the footer rule

\setlength\parindent{0pt} % Removes all indentation from paragraphs

%----------------------------------------------------------------------------------------
%	CODE INCLUSION CONFIGURATION
%----------------------------------------------------------------------------------------

\definecolor{MyDarkGreen}{rgb}{0.0,0.4,0.0} % This is the color used for comments
\lstloadlanguages{Perl} % Load Perl syntax for listings, for a list of other languages supported see: ftp://ftp.tex.ac.uk/tex-archive/macros/latex/contrib/listings/listings.pdf
\lstset{language=Perl, % Use Perl in this example
        frame=single, % Single frame around code
        basicstyle=\small\ttfamily, % Use small true type font
        keywordstyle=[1]\color{Blue}\bf, % Perl functions bold and blue
        keywordstyle=[2]\color{Purple}, % Perl function arguments purple
        keywordstyle=[3]\color{Blue}\underbar, % Custom functions underlined and blue
        identifierstyle=, % Nothing special about identifiers                                         
        commentstyle=\usefont{T1}{pcr}{m}{sl}\color{MyDarkGreen}\small, % Comments small dark green courier font
        stringstyle=\color{Purple}, % Strings are purple
        showstringspaces=false, % Don't put marks in string spaces
        tabsize=5, % 5 spaces per tab
        %
        % Put standard Perl functions not included in the default language here
        morekeywords={rand},
        %
        % Put Perl function parameters here
        morekeywords=[2]{on, off, interp},
        %
        % Put user defined functions here
        morekeywords=[3]{test},
       	%
        morecomment=[l][\color{Blue}]{...}, % Line continuation (...) like blue comment
        numbers=left, % Line numbers on left
        firstnumber=1, % Line numbers start with line 1
        numberstyle=\tiny\color{Blue}, % Line numbers are blue and small
        stepnumber=5 % Line numbers go in steps of 5
}

% Creates a new command to include a perl script, the first parameter is the filename of the script (without .pl), the second parameter is the caption
\newcommand{\perlscript}[2]{
\begin{itemize}
\item[]\lstinputlisting[caption=#2,label=#1]{#1.pl}
\end{itemize}
}

%----------------------------------------------------------------------------------------
%	DOCUMENT STRUCTURE COMMANDS
%	Skip this unless you know what you're doing
%----------------------------------------------------------------------------------------

% Header and footer for when a page split occurs within a problem environment
\newcommand{\enterProblemHeader}[1]{
\nobreak\extramarks{#1}{#1 continued on next page\ldots}\nobreak
\nobreak\extramarks{#1 (continued)}{#1 continued on next page\ldots}\nobreak
}

% Header and footer for when a page split occurs between problem environments
\newcommand{\exitProblemHeader}[1]{
\nobreak\extramarks{#1 (continued)}{#1 continued on next page\ldots}\nobreak
\nobreak\extramarks{#1}{}\nobreak
}

\setcounter{secnumdepth}{0} % Removes default section numbers
\newcounter{homeworkProblemCounter} % Creates a counter to keep track of the number of problems

\newcommand{\homeworkProblemName}{}
\newenvironment{homeworkProblem}[1][Problem \arabic{homeworkProblemCounter}]{ % Makes a new environment called homeworkProblem which takes 1 argument (custom name) but the default is "Problem #"
\stepcounter{homeworkProblemCounter} % Increase counter for number of problems
\renewcommand{\homeworkProblemName}{#1} % Assign \homeworkProblemName the name of the problem
\section{\homeworkProblemName} % Make a section in the document with the custom problem count
\enterProblemHeader{\homeworkProblemName} % Header and footer within the environment
}{
\exitProblemHeader{\homeworkProblemName} % Header and footer after the environment
}

\newcommand{\problemAnswer}[1]{ % Defines the problem answer command with the content as the only argument
\noindent\framebox[\columnwidth][c]{\begin{minipage}{0.98\columnwidth}#1\end{minipage}} % Makes the box around the problem answer and puts the content inside
}

\newcommand{\homeworkSectionName}{}
\newenvironment{homeworkSection}[1]{ % New environment for sections within homework problems, takes 1 argument - the name of the section
\renewcommand{\homeworkSectionName}{#1} % Assign \homeworkSectionName to the name of the section from the environment argument
\subsection{\homeworkSectionName} % Make a subsection with the custom name of the subsection
\enterProblemHeader{\homeworkProblemName\ [\homeworkSectionName]} % Header and footer within the environment
}{
\enterProblemHeader{\homeworkProblemName} % Header and footer after the environment
}

%----------------------------------------------------------------------------------------
%	NAME AND CLASS SECTION
%----------------------------------------------------------------------------------------

\newcommand{\hmwkTitle}{Assignment\ \#2} % Assignment title
\newcommand{\hmwkDueDate}{Thursday,\ September 30th,\ 2014} % Due date
\newcommand{\hmwkClass}{CS 381} % Course/class
\newcommand{\hmwkClassTime}{12:00pm} % Class/lecture time
\newcommand{\hmwkClassInstructor}{Prof. Grigorescu} % Teacher/lecturer
\newcommand{\hmwkAuthorName}{Yao Xiao(xiao67)} % Your name

%----------------------------------------------------------------------------------------
%	TITLE PAGE
%----------------------------------------------------------------------------------------

\title{
\vspace{2in}
\textmd{\textbf{\hmwkClass:\ \hmwkTitle}}\\
\normalsize\vspace{0.1in}\small{Due\ on\ \hmwkDueDate}\\
\vspace{0.1in}\large{\textit{\hmwkClassInstructor\ \hmwkClassTime}}
\vspace{3in}
}

\author{\textbf{\hmwkAuthorName}}
\date{} % Insert date here if you want it to appear below your name

%----------------------------------------------------------------------------------------
 \everymath{\displaystyle} 
\begin{document}

\maketitle

%----------------------------------------------------------------------------------------
%	TABLE OF CONTENTS
%----------------------------------------------------------------------------------------

\setcounter{tocdepth}{1} % Uncomment this line if you don't want subsections listed in the ToC

\newpage
%\tableofcontents
\newpage

%----------------------------------------------------------------------------------------
%	PROBLEM 1
%----------------------------------------------------------------------------------------

% To have just one problem per page, simply put a \clearpage after each problem

\begin{homeworkProblem}
Give the solution to the following recurrences by applying the Master theorem when possible,
or by any other of the methods we’ve learned when the Master theorem doesn’t apply. Assume
T (1) = 1. \\
1. T (n) = T (n/5) + T (7n/10) + 1 \\
2. T (n) = 2T (n - 1) + 5 \\
3. T (n) = 6T (2n/5) + log 2 n \\
4. T (n) = 2T (n/2) + $n^{1.4}$ \\
5. T (n) = 12T (n/12) + 11n \\
6. T (n) = T ( $\sqrt{n}$) + 12. \\
\problemAnswer{
$1.T(n)=T(n/5)+T(7n/10)$ We can't use master therom. 
Assume $T(n)<=cn $ \\
So $T(n)<=1/5cn+7/10cn+1$\\
=$0.9cn+1$\\
=$cn-0.9cn+1$\\
$<=cn$\\
So $T(n)=\Theta(n)$
\\ \\
$2. T(n)=2T(n-1)+5 = 4T(n-2)+10 = 8T(n-3)+20 = 2^k*T(n-k)+5^{2^{k-1}}$\\ \\
$when \ k=n-1 \ T(n)=2^{n-1}*T(1)+5^{2^{n-2}}=2^{n-1}+5^{2^{n-2}}  $ \\ \\
$3.T(n) = a \; T\!\left(\frac{n}{b}\right) + f(n)  \;\;\;\; \mbox{where} \;\; a \geq 1 \mbox{, } b > 1 $ \\
$ a=6 \\ b=5/2 \\ f(n)=log_{2}{n}\\ c=log_{5/2}6 \approx 1.95 \\ f(n)=\Omega(n^c)	  \\
 T(n)=\Theta(log_{2}n)$
\\ \\
$4. a=2 \\ b=2 \\ f(n)=n^{1.4} log_2{2}=1<1.4 \\ So T(n)=\Theta(n^{1.4}) $ \\ \\
$5. a=12 \\ b=12 \\ f(n)=11n \\ log_{12}{12}=1=c \\ So T(n)=\Theta(nlogn )$ \\ \\
$6. T(n)=T(\sqrt{n})+O(1) \\ Let m=log_{2}n n=2^m$ \\
$ T(2^m)=T(2^{m/2})+O(1)$ \\
$ Set K(m)=T(2^m), \ K(m)=K(m/2)+O(1), \ where a=1 b=2 $ \\
$ K(m)=\Theta(log2(m)) T(n)=\Theta(log2(log2(n))) $\\




}
\end{homeworkProblem} 
%----------------------------------------------------------------------------------------
%	PROBLEM 2
%----------------------------------------------------------------------------------------

\begin{homeworkProblem}

Dan and Alex play the following game: Dan picks an integer from 1 to m and Alex is trying to
figure it out by asking Dan as few questions as possible. Dan is willing to respond truthfully, but
he announces that he will only answer ‘yes’ or ‘no’ to any of Alex’s questions. Design a strategy
such that Alex can always guess the number using only o(m) questions

\problemAnswer{
Ask every time if the number is bigger than (upperbound+lowerbound)/2, and divide change the upper bound and lower bound based on the answer.
}


\end{homeworkProblem}

%----------------------------------------------------------------------------------------
%	PROBLEM 3
%----------------------------------------------------------------------------------------

\begin{homeworkProblem}

You are given two n-digit numbers a, b > 0, and have to come up with a fast algorithm to
multiply them. \\
1. What is a trivial upper bound on the number of steps it takes to compute the product ab ? \\
2. Design a divide-and-conquer procedure that uses O(n1.58 ) digit-operations. \\

\problemAnswer{
1. The upper bound will be $O(n^2)$ since you have to multiple by each number. \\
2. Set $A=x1*10^{n}+x0 \ B=y1*10^{n}+y0 $ \\
$ So \ AB=x1y1*10^{2n}+(x1y0+x0y1)10^n+x0y0 $ \\
$ where (x1y0+x0y1)=(x1+x0)(y1+y0)-x1y1-x0y0 $ \\
So we just need to compute for every step
$ x0y0 x1y1 (x1+x0)(y1+y0) $

there are three component and each time we divide the number by half \\ \\

The formula will be $ T(n)=3(t(n/2))+c$ \\ \\
The master theorem gives $T(n)=\Theta(log_{2}3)$ \\


\textit{Source: Piazza post @47}
}
\end{homeworkProblem}

\begin{homeworkProblem}

\problemAnswer{
1. The time would become 
$ T(n)=T(\ceil{n/11})+T(9/10)+\Theta(n)=1.11cn+\Theta(n) $


2. The time would become 
$ T(n)=T(\ceil{n/3})+T(5/6n)+\Theta(n)\neq cn+\Theta(n) $

Since after selecting, the element which could be abandoned is only $1/6$ of the total elements

}
\end{homeworkProblem}
%----------------------------------------------------------------------------------------

\begin{homeworkProblem}
 I don't know , but 
$http://en.wikipedia.org/wiki/Closes\_pair\_of\_points\_problem$

\end{homeworkProblem}
\end{document}